\documentclass{article}

\usepackage[utf8]{inputenc}
\usepackage{fullpage}
\usepackage{times}
\usepackage{tcolorbox}
\usepackage{enumitem}
\usepackage{multicol}
\usepackage{bookmark}
\usepackage{CJKutf8}
\usepackage[normalem]{ulem}

\bookmarksetup{
  numbered, 
  open,
}

\setlist{itemsep=2pt}
\renewcommand{\thesection}{Lecture \arabic{section}}
\renewcommand{\thesubsection}{\arabic{section}.\arabic{subsection}}
%\renewcommand{\thesubsubsection}{\indent\arabic{section}.\arabic{subsection}.\arabic{subsubsection}}
\setcounter{section}{9}

\begin{document}
 \begin{CJK}{UTF8}{min}

\noindent
{JAPAN 101 \hfill Hao Pan}\\
{Shimoda, Fumie}\\
{Spring 2018}

%%%%%%%%%%%%%%%%%%%%%%%%%%%%%%%%%%%

\begin{center}
\section{}
\noindent
{\hfill 11/07/2018 [W]}
\end{center}

\subsection{Te Form Verbs}

\bigskip

Used when (page 150):
\begin{itemize}
\item Making requests.
\item Giving and asking for permission.
\item Stating that something is forbidden.
\item Forming a sentence that describes 2 events/activities.
\end{itemize}

\subsubsection{Group 1}
General Rule: Replace last character of verb following a set of rules.

\bigskip

\begin{multicols}{2}

\begin{tabular}{ | l | l | l | }
\hline
Base & Te Form & English Meaning\\
\hline
あう & あって & to meet\\
たつ & たって & to stand up\\
はいる & はいって & to enter\\
あそふ & あそんで & to play\\
やすむ & やすんで & to be absent/to rest\\
しぬ & しんで & to die\\
きく & きいて & to listen\\
いそぐ & いそいで & to hurry\\
けす & けして & to turn off/erase\\
いく & いって & to go\\
\hline
\end{tabular}

\vfill\null
\columnbreak

\begin{tabular}{ | l @{${}\rightarrow{}$} l | }
\hline
\multicolumn{2}{ | c | }{Rules}\\
\hline
う、つ、る & って\\
ふ、む、ぬ & んで\\
く & いて\\
ぐ & いで\\
す & して\\
\hline
くる & きて\\
\hline
れいがい & いって\\
\hline
\end{tabular}

\end{multicols}

\subsubsection{Group 2}
General Rule: Remove last character (る) and add て.

\bigskip

\begin{tabular}{ | l | l | l | }
\hline
Base & Te Form & English Meaning\\
\hline
あける & あけて & to open\\
しめる & しめて & to close\\
おりる & おりて & to get off\\
かりる & かりて & to borrow\\
\hline
\end{tabular}

\subsubsection{Group 3}
General Rule: する becomes して, and くる becomes きて. Every other group 3 verb is a prefix of these 2.

\bigskip

\begin{tabular}{ | l | l | l | }
\hline
Base & Te Form & English Meaning\\
\hline
する & して & to do\\
べんきょうする & べんきょうして & to study\\
くる & きて & to come\\
つれてくる & つれてきて & to bring (a person)\\
もってくる & もってきて & to bring (an object)\\
\hline
\end{tabular}

%%%%%%%%%%%%%%%%%%%%%%%%%%%%%%%%%%%
\subsection{Polite Request}
[v-te form]ください。

\begin{itemize}
\item {\bf Ex}. \uline{きょうかしょ}を\uline{あけて}ください。 | \emph{Please \uline{open} the \uline{textbook}.}
\end{itemize}

%%%%%%%%%%%%%%%%%%%%%%%%%%%%%%%%%%%
\subsection{Giving Permission}
[v-te form]もいいです。

\begin{itemize}
\item {\bf Ex}. \uline{おふる}に\uline{はいって}もいいです。 | \emph{You may \uline{take} a \uline{shower}.}
\item To ask for permission, add か at the end of a sentence.
\end{itemize}

%%%%%%%%%%%%%%%%%%%%%%%%%%%%%%%%%%%
\subsection{Something Forbidden}
[v-te form]はいけません。

\begin{itemize}
\item {\bf Ex}. \uline{としょかん}でたべものを\uline{たべて}はいけません。 | \emph{You must not \uline{eat} food in the \uline{library}.}
\end{itemize}












\end{CJK}
\end{document}