\documentclass{article}

\usepackage[utf8]{inputenc}
\usepackage{fullpage}
\usepackage{times}
\usepackage{tcolorbox}
\usepackage{enumitem}

\setlist{itemsep=2pt}

\begin{document}

\noindent
{JAPAN 101 \hfill Hao Pan}\\
{Shimoda, Fumie}\\
{Spring 2018}

%%%%%%%%%%%%%%%%%%%%%%%%%%%%%%%%%%%

\begin{center}
\underline{\bf \large Lecture 1}\\
\noindent
{\hfill 09/05/2018 [W]}
\end{center}

\underline{\bf \large Numbers}\\

%\begin{tcolorbox}[colback=green!35!white]
\begin{tabular}{ r l || r l || r l || r l }
%\hline
0 & zero & 5 & go & 10 & juu & 100 & kyaku\\
1 & ichi & 6 & roku & 20 & nijuu &&\\
2 & ni & 7 & nana/shichi & 30 & sanjuu &&\\
3 & san & 8 & hachi & 40 & yonjuu &&\\
4 & yon/shi & 9 & kyuu & 50 & gojuu &&
%\hline
\end{tabular}\\
%\end{tcolorbox}

\rule{6in}{0.4pt}\\

%\begin{tcolorbox}
\underline{\bf Time}\\

\begin{tabular}{ r l | r l | r l }
\multicolumn{2}{ c | }{Hour} & \multicolumn{4}{ | c }{Minute}\\
\hline
\multicolumn{2}{ c | }{\{number\}-ji} & \multicolumn{2}{ | c | }{\{number\}-fun} & \multicolumn{2}{ | c }{\{number\}-pun}\\
4:00 & yo-ji & :05 & go-fun & :10 & juu-pun\\
7:00 & shichi-ji & :15 & juu-go-fun & :20 & ni-jup-pun\\
9:00 & ku-ji & :25 & ni-juu-go-fun & :30 & san-jup-pun/han\\
&& :35 & san-juu-go-fun & :40 & yon-jup-pun
\end{tabular}

\begin{itemize}
\item am = GOZEN
\item pm = GOGO
\end{itemize}

{\bf Ex.} {\it gozen yo-ji han | 4:30am}\\
%\end{tcolorbox}

\rule{6in}{0.4pt}\\

\underline{\bf Age}

\{number\}-sai\\

\underline{Exceptions:}

\begin{tabular}{ r l }
1 & i-sai\\
8 & ha-sai\\
10 & ju-sai\\
20 & hatachi\\
\end{tabular}

{\bf Ex.} {\it Watashi wa hatachi desu | My age is 20}\\

\rule{6in}{0.4pt}\\

\underline{\bf Phone Number}

{\it denwa bangoo}\\

{\bf Ex.}\\
\indent roku yon nana no ni roku ni no roku roku roku san desu\\
\indent 6 4 7 - 2 6 2 - 6 6 6 3\\

%\rule{6in}{0.4pt}\\
\newpage

\underline{\bf Academic Year}

{\it \{number\}-nensee}\\

4th year | yo-nensee\\

\rule{6in}{0.4pt}\\

\underline{\bf \large Question Sentence}

{\it Add "ka"}\\

\{Noun1\} wa \{Noun2\} desu ka.

{\bf Answer} using {\bf hai} or {\bf ee} (before the sentence) to mean {\bf yes}.

{\bf Answer} using {iie} to mean {\bf no}.\\

\begin{tcolorbox}[colback=blue!10!white]
{\bf Examples:}

\begin{tabular}{ l | l }
Anata wa gakusee desu ka. & Are you a student?\\
\hline
\hline
Hai. Gakusee desu. & Yes. I am a student.\\
Hai. Soo desu. & Yes.\\
Iie. Sensee desu. & No. I am a teacher.\\
\end{tabular}

\begin{itemize}
\item {\it Soo} is a substitute for a noun that has already been mentioned.
\item {\bf Note:} State correct noun if you answer no.
\end{itemize}
\end{tcolorbox}

\begin{tcolorbox}[colback=yellow!40!white]
\underline{\bf Common Questions}

\begin{tabular}{ l | l }
Denwa bangoo wa nan desu ka. & What is your telephone number?\\
Ima nanji desu ka. & What time is it now?\\
Nan-sai desu ka. & What is your age?\\
Oikutsu desu ka. & What is your age? (polite)\\
Nan-nensee desu ka. & What academic year are you in?\\
\end{tabular}

\begin{itemize}
\item nan | what (in question context).
\item ima | now.
\end{itemize}
\end{tcolorbox}


\end{document}