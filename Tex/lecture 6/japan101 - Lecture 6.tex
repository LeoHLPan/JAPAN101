\documentclass{article}

\usepackage[utf8]{inputenc}
\usepackage{fullpage}
\usepackage{times}
\usepackage{tcolorbox}
\usepackage{enumitem}
\usepackage{bookmark}
\usepackage{CJKutf8}

\bookmarksetup{
  numbered, 
  open,
}

\setlist{itemsep=2pt}
\renewcommand{\thesection}{Lecture \arabic{section}}
\renewcommand{\thesubsection}{\arabic{section}.\arabic{subsection}}
\setcounter{section}{5}

\begin{document}
 \begin{CJK}{UTF8}{min}

\noindent
{JAPAN 101 \hfill Hao Pan}\\
{Shimoda, Fumie}\\
{Spring 2018}

%%%%%%%%%%%%%%%%%%%%%%%%%%%%%%%%%%%

\begin{center}
\section{}
\noindent
{\hfill 06/06/2018 [W]}
\end{center}

\subsection{Frequency Adverbs}

\begin{tabular}{ l | l || l }
Japanese & English & Matching Sentence\\
\hline
まいにち & Everyday & -\\
いつも & Always & あさごはんをたべます。\\
よく & Often & I eat breakfast.\\
ときどき & Sometimes & -\\
\hline
あまり & Not often & あさごはんをたべません。\\
めったに & Seldom & I eat breakfast.\\
ぜんぜん & Never & -\\
\end{tabular}

\begin{itemize}
\item Frequency adverbs are placed in front of a sentence.
\item {\bf Note}: "あさごはんをたべません" normally means "I do not eat breakfast", but in the context of the negative frequency adverbs, it becomes a positive sentence.
\item {\bf Examples}
\begin{itemize}
\item {\bf Q}: コーヒー(こうひい)をのみますか。 | Do you drink coffee?
\item {\bf A}: はい、__のみます。 | Yes, I drink \_\_.
\item {\bf A}: いいえ、__のみません。 | No, I \_\_ drink.
\end{itemize}
\end{itemize}

\subsection{Time Reference}

[Numerical Time Expression] に (Obj. を) v.masu form ます。

\begin{itemize}
\item {\bf Ex.} じゅうじにおきます。 | I wake up at 10-o-clock.
\item {\bf Questions}
\begin{itemize}
\item なんじにおきますか。 | What time do you wake up?
\item \underline{はちがつ}にテニス(てにす)をします。 | I will play tennis in \underline{August}.
\item \underline{にちようび}にくつをかいます。 | I will buy shoes on \underline{Sunday}.
\end{itemize}
\item To show {\bf approximation}, replace に with ごろ.
\begin{itemize}
\item くじごろおkます。 | I wake up at about 9.
\end{itemize}
\end{itemize}

\clearpage

\subsection{Relative Time}

\begin{itemize}
\item {\bf Ex.} \underline{あした}にほんごをべんきょうします。 | I will study Japanese \underline{tomorrow}.
\item {\bf Note}: Since "tomorrow" is relative to present time, に is not used.
\end{itemize}

\underline{\bf Examples}

\begin{itemize}
\item \underline{らいしゅう}えいがをみます。 | I will watch movie \underline{next week}.
\item For more examples, consider the time words on textbook page 127.
\end{itemize}

\subsection{Parts of the Day}

\begin{itemize}
\item \underline{あさ}(に)コーヒーをのみます。 | I drink coffee in the \underline{morning}.
\item \underline{じゅうまつ}(に)テレビ(てれび)をみます。 | I watch TV at \underline{the weekend}.
\end{itemize}

\subsection{Invitation}

[Time] に (Obj. を) v.masu-form ませんか。

\begin{itemize}
\item {\bf Q}: コーヒーをのみませんか。 | Would you like to drink coffee?
\item {\bf A}: いいですね。 | Friendly acceptance.
\item {\bf A}: すみませんか、ちょって… | Sorry, I'm a little... (politely decline).
\end{itemize}







\end{CJK}
\end{document}