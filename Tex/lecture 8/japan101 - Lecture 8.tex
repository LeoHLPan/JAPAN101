\documentclass{article}

\usepackage[utf8]{inputenc}
\usepackage{fullpage}
\usepackage{times}
\usepackage{tcolorbox}
\usepackage{enumitem}
\usepackage{multicol}
\usepackage{bookmark}
\usepackage{CJKutf8}
\usepackage[normalem]{ulem}

\bookmarksetup{
  numbered, 
  open,
}

\setlist{itemsep=2pt}
\renewcommand{\thesection}{Lecture \arabic{section}}
\renewcommand{\thesubsection}{\arabic{section}.\arabic{subsection}}
%\renewcommand{\thesubsubsection}{\indent\arabic{section}.\arabic{subsection}.\arabic{subsubsection}}
\setcounter{section}{7}

\begin{document}
 \begin{CJK}{UTF8}{min}

\noindent
{JAPAN 101 \hfill Hao Pan}\\
{Shimoda, Fumie}\\
{Spring 2018}

%%%%%%%%%%%%%%%%%%%%%%%%%%%%%%%%%%%

\begin{center}
\section{}
\noindent
{\hfill 27/06/2018 [W]}
\end{center}

\subsection{Duration}

__がん $\rightarrow$ \_\_ (duration)

\begin{itemize}
\item \uline{ろくじ}かんねました。 | \emph{I slept for \uline{6 hours}.}
\item \uline{ろくじかん}\uline{ぐらい}ねました。 | \emph{I slept for \uline{about} \uline{6 hours}.}
\item Add か to the end to make it a question.
\end{itemize}

%%%%%%%%%%%%%%%%%%%%%%%%%%%%%%%%%%%
\subsection{Usage of と}

\subsubsection{Connecting 2 Nouns}
\begin{itemize}
\item きって\uline{と}えんぴつをかいました。 | \emph{I bought stamps \uline{and} pencils.}
\end{itemize}

\subsubsection{Describe with Whom}
\begin{itemize}
\item ともだち\uline{も}テニスをしました。 | \emph{I played tennis \uline{with} a friend.}
\item {\bf Q}: だれとテニスをしました。 | \emph{Who did you play tennis with?}
\begin{itemize}
\item {\bf A}: \uline{ひとりで}しました。 | \emph{I played \uline{alone}.}
\end{itemize}
\end{itemize}

%%%%%%%%%%%%%%%%%%%%%%%%%%%%%%%%%%%
\subsection{Usage of も (Also)}

\subsubsection{Noun Comparison}
\begin{itemize}
\item メアリー(Mary)はがくせいです。 | \emph{Mary is a student.}
\item たけし\uline{も}がくせいです。 | \emph{Takeshi is \uline{also} a student.}
\end{itemize}

\subsubsection{Verb Activity}
\begin{itemize}
\item さかなをたべます。 | \emph{I eat fish.}
\item にく\uline{も}たべます。 | \emph{I \uline{also} eat meat.}
\end{itemize}

\subsubsection{Goal of Movement}
\begin{itemize}
\item きょうと[に/へ]いきます。 | \emph{I will go to Kyoto.}
\item とうきょう\uline{[に/へ]も}いきます。 | \emph{I will \uline{also} go to Tokyo.}
\end{itemize}

\subsubsection{Time Reference}
\begin{itemize}
\item すいようびににほんごをべんきょうします。 | \emph{I study Japanese on Wednesday.}
\item もくようび\uline{にも}にほんごをべんきょうします。 | \emph{I \uline{also} study Japanese on Thursday.}
\item {\bf Note}: に is not needed for relative time reference.
\end{itemize}

\subsubsection{Locations}
\begin{itemize}
\item がっこうでにほんごをはなします。 | \emph{I study Japanese at school.}
\item うち\uline{でも}にほんごをはなします。 | \emph{I \uline{also} study Japanese at home.}
\end{itemize}

%%%%%%%%%%%%%%%%%%%%%%%%%%%%%%%%%%%
\subsection{Adjectives}
\vspace{-3mm}
\emph{See handout for list of adjectives.}

\subsubsection{I-Adjectives}
[N]は
$\left\{
\begin{tabular}{@{} ll @{}}
[I-adj]&です。 \\
{[I-adj w/o い]}&くないです。\\
\end{tabular}
\right.$

\begin{itemize}
\item にほんごは\uline{あもしろい}です。 | \emph{Japanese is \uline{interesting}.}
\item にほんごは\uline{あもしろくない}です。 | \emph{Japanese is \uline{not interesting}.}
\item {\bf Q}: にほんごはあもしろいですか。 | \emph{Is Japanese interesting?}
\begin{itemize}
\item {\bf A}: はい、\uline{あもしろい}です。 | \emph{Yes, it's \uline{interesting}.}
\item {\bf A}: いいえ、\uline{あもしろ}くないです。 | \emph{No, it's not \uline{interesting}.}
\end{itemize}
\end{itemize}

\subsubsection{NA-Adjectives}
[N]は
$\left\{
\begin{tabular}{@{} ll @{}}
{[Na-adj]}&です。\\
{[Na-adj]}&じゃないです。\\
\end{tabular}
\right.$

\begin{itemize}
\item にほんごは\uline{かんたん}です。 | \emph{Japanese is \uline{simple}.}
\item にほんごは\uline{かんたんじゃない}です。 | \emph{Japanese is \uline{not simple}.}
\item {\bf Q}: にほんごはかんたんですか。 | \emph{Is Japanese simple?}
\begin{itemize}
\item はい、\uline{かんたん}です。 | \emph{Yes, it's \uline{simple}.}
\item いいえ、\uline{かんたん}じゃないです。 | \emph{No, it's not \uline{simple}.}
\end{itemize}
\end{itemize}

%%%%%%%%%%%%%%%%%%%%%%%%%%%%%%%%%%%
\subsection{Attaching Adjectives to Nouns}

\begin{itemize}
\item To attach adjectives to nouns, simply add them before the noun.
\item Na-adjectives need な in between it and the noun.
\item {\bf Examples}:
\begin{itemize}
\item たかいでんわ | \emph{Expensive phone}
\item きれいなへや | \emph{Beautiful/clean room}
\end{itemize}
\item {\bf Q}: どんな[N]ですか。 | \emph{What kind of [N]?}
\begin{itemize}
\item {\bf A}: [Adj] (な) [N]です。
\end{itemize}
\end{itemize}










\end{CJK}
\end{document}